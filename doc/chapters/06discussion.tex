\documentclass[main.tex]{subfiles}
\begin{document}
\chapter{Discussion}\label{chap:discussion}

This thesis started out with two research questions: First: can a 3D CNN be trained to perform the task of nodule detection and second, how can it's learned features be extracted and used for understanding the solution? 

To the first question: it was shown that it is possible to train a network with promising results upon which further optimization could be applied. The performance of the learned network was at XXX $\%$ which is surprising for it's simple structure. With more time and computational power it seems very possible to increase the performance further. Ways forward could include more samples from the database and a richer augmentation of the samples which allow for training even deeper networks. Other network parameters could easily be systematically varied and tested for effectiveness, allowing for further optimization.

The code for this thesis is openly available on \hyperref[GitHub]{''https://github.com/AndreaSuckro/acts''}. 

\section{Items of future code optimization}
\begin{description}
\item[cluster parameters] are at the moment not completely adapted to the optimal need of the learning process and could be further improved.
\item[input pipeline] the input pipeline is not using queues as suggested in the TensorFlow documentation, but just reads in all the lung ct data.
\item[preprocessing] further steps in the preprocessing could improve the results, for example a true 3d rotation and different scales of the same patch could be used. The ratio of healthy and nodule patches is at the moment way to high ($50:50$). This could be adjusted to a more reasonable value although this may prolong the training phase.
\item[Training time] due to the way the institute set up the grid, all users have a walltime which stops training after several hours. One could pick up the training after being shut down from the latest checkpoint.
\end{description}

\section{Items of future investigation}
\begin{description}
\item[Fully convolutional network] to eliminate the influence on the performance of the dense layers and keep the location information until the end.
\item[Optimizers] it would have been nice to check for the influence of different optimizers for this problem.
\item[Minimal Architecture] how can the number of layers and/or neurons be reduced. What is the optimal distribution for the filter sizes to be used.
\item[Metrics of one patient] For one patient it would be interesting to see how the network fairs when one cube after the other is selected and fed in to the network. How does the FP rate look for example?
\end{description}

% the end :)
\end{document}
