\graphicspath{ {imgs/} }
\documentclass[../Thesis.tex]{subfiles}
\begin{document}
\chapter{Dataset}
The Lung Image Database Consortium image collection (LIDC-IDRI) consists of diagnostic and lung cancer screening thoracic computed tomography (CT) scans with marked-up annotated lesions. It is a web-accessible international resource for development, training, and evaluation of computer-assisted diagnostic (CAD) methods for lung cancer detection and diagnosis. Initiated by the National Cancer Institute (NCI), further advanced by the Foundation for the National Institutes of Health (FNIH), and accompanied by the Food and Drug Administration (FDA) through active participation, this public-private partnership demonstrates the success of a consortium founded on a consensus-based process.
Seven academic centers and eight medical imaging companies collaborated to create this data set which contains 1018 cases.  Each subject includes images from a clinical thoracic CT scan and an associated XML file that records the results of a two-phase image annotation process performed by four experienced thoracic radiologists. In the initial blinded-read phase, each radiologist independently reviewed each CT scan and marked lesions belonging to one of three categories ("nodule > or =3 mm," "nodule <3 mm," and "non-nodule > or =3 mm"). In the subsequent unblinded-read phase, each radiologist independently reviewed their own marks along with the anonymized marks of the three other radiologists to render a final opinion. The goal of this process was to identify as completely as possible all lung nodules in each CT scan without requiring forced consensus. \cite{armato2011lung}

\section{Content and Structure}
The dataset contains a folder for each patient. These folders contain a full Chest CT scan and the annotations done by the radiologists. The CT Scan is encoded in a list of .dicom files and the annotations as a .xml file.

i.	Reconstruction interval and collimation both < 3 mm
i.	Scans may include high levels of noise or streak, motion, or metal artifacts (this will be characterized in the assessment of image quality)
i.	Other pathology may be present, unless it is spatially contiguous with nodules and substantially interferes with their visual interpretation

Nominated cases could have 0 to 6 nodules, where

i.	Nodules may represent primary lung cancers, metastatic disease, or non-cancerous processes


\subsection{Annotation Structure}
Two different types of nodules are encoded in the data: nodules with a diameter of $>=$ 3mm and nodules smaller than that. The big nodules have extensive information stored with them: a rich edge map which outlines a complete contour for them in all sections \ref{fig:bigNod}. They also have a measure for their characteristics (like a measure for their subtlety and malignancy on a scale from 1 to 5). Those extra information have not been used in the learning process for this thesis.

\lstset{
  language=XML,
  morekeywords={characteristics,noduleID,edgeMap,imageZposition}
}
\begin{figure}
\begin{lstlisting}
      <noduleID>IL057_127581</noduleID>
      <characteristics>
        <subtlety>4</subtlety>
        <malignancy>3</malignancy>
        [...]
      </characteristics>
      
      <edgeMap>
        <xCoord>103</xCoord>
        <yCoord>391</yCoord>
      </edgeMap>
 
      <imageZposition>-232.535004</imageZposition>
       
      <edgeMap>
         <xCoord>104</xCoord>
         <yCoord>393</yCoord>
      </edgeMap>
\end{lstlisting}
\caption{A shortened example xml annotation for a nodule with diameter $>=$ 3mm.}
\label{fig:bigNod}
\end{figure}

Nodules with a smaller diameter have less information stored with them. They only contain the approximate center of mass for the nodule \ref{fig:smallNod}.

\begin{figure}
\begin{lstlisting}
	<noduleID>7</noduleID>
	<roi>
	<imageZposition>-227.535004</imageZposition>
        <imageSOP_UID>1.3.6.1.4...</imageSOP_UID>
	<inclusion>TRUE</inclusion>
	<edgeMap>
	   <xCoord>127</xCoord>
	   <yCoord>370</yCoord>
	</edgeMap>
	</roi>
\end{lstlisting}
\caption{Nodules with a diameter of $<$ 3mm have only the center of mass stored.}
\label{fig:smallNod}
\end{figure}

\section{Properties of the Data}

\section{Preprocessing}
Some of the extra folders for a patient contain only a few scans and not a complete CT. Those folders were ignored. The other scanned files are read in and sliced up for the network.

The network is working on cubical sections of the total ct scan. The cubes are picked at random
\subsection{Splitting the dataset}
The whole dataset was split into a training, testing and validation dataset. The training dataset is used during the learning process of the neural network.

The testing dataset is used to measure the performance of the network while it is trained on the train dataset.

\end{document}
