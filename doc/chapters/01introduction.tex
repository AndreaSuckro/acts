\documentclass[main.tex]{subfiles}
\begin{document}
\chapter{Introduction}
General information about the use case and for which illnesses this becomes important, how many scans are made
\section{Overview}
 A pulmonary nodule is a small, round (parenchymal nodule) or worm (juxtapleural nodule) shaped lesion in the lungs
 ground glass opacity nodules.
 GGO nodules are more likely to be malignant than solid nodules
\section{Current Medical Approach}
Requirement to analyse X pictures per patient
current programs (screenshot OsiriX)
fallacy rates 
\section{Opportunities for Assistance}
unnecessary scans can be avoided, improve error rate of the radiologists

Despite much effort being devoted to the computer-aided nodule detection problem, lung CAD systems remain an ongoing
research topic [18]. One of the major difficulties is the detection of GGO nodules with low-dose thin-slice CT screening. Another two difficulties are the detection of nodules that are adjacent to vessels or the chest wall when they have very similar intensity; and the detection of nodules that are nonspherical in shape. In such cases, intensity thresholding or model-based methods might fail to identify those nodules.

\end{document}